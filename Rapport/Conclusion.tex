Le projet a \'{e}t\'{e} réussi dans l'ensemble et se cloture dans le temps qui nous etait imparti. \\

N\'{e}anmoins, nous avons eu quelques petites difficultés nottament au moment de la conception de nos TADs. Dans un premier temps, nous avons mal saisi l'utilité du TAD Octet, et avons fait fausse route en impl\'{e}mentant un tableau contenant le type Bit. Nous avons alors chang\'{e} de route en comprenant que la lecture et l'\'{e}criture d'octets dans des fichiers en C se faisait avec un unsigned char.

Un autre ennui au moment de la conception detaill\'{e}e : dans les proc\'{e}dures de codage et decodage, nous avons voulu concatener les codes binaires pour n'en former plus qu'un seul qui sera ensuite \'{e}crit dans le fichier compress\'{e}, ou bien decodé puis \'{e}crit dans le fichier d\'{e}compress\'{e}. Or pendant la phase de d\'{e}veloppement, nous nous sommes rendus compte qu'à l'ex\'{e}cution la proc\'{e}dure de concat\'{e}nation plantait pour des fichiers de taille sup\'{e}rieure de 2\^{}15 octets. Nous avons donc decid\'{e} de coder et de decoder les octets un par un.

Malgr\'{e} la r\'{e}ussite du projet il y a des points que nous aurions pu am\'{e}liorer. Nottament l'impl\'{e}mentation d'un buffer pour le type FichierBinaire. En effet, la lecture et ecriture d'octet se fait un par un dans le fichier, le temps d'ex\'{e}cution est donc long pour des fichiers volumineux. Pour exemple, un fichier de 10Mo est compress\'{e} en 10 secondes.

Pour conclure donc, nous avons bien respect\'{e} la m\'{e}thodologie du cours et les diff\'{e}rentes phase explicit\'{e}e dans l'\'{e}nonc\'{e}. Ce fut un projet formateur en vue des futurs PICs.
