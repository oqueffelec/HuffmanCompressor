% auteur : Wacyk Jean-Gabriel

\begin{algorithme}
  \fonction{getOctet}{o : Octet}{\naturel}
  {
  \retourner{o.octet}
  }
\end{algorithme}

\begin{algorithme}
  \fonction{nombreBit}{o : Octet}{\naturel}
  {
  \retourner{o.nb}
  }
\end{algorithme}

\begin{algorithme}
  \fonction{octetZero}{}{Octet}
  {
  o : Octet
  }
  {
  \affecter{o.octet}{0}
  \affecter{o.nb}{0}
  \retourner{o}
  }
\end{algorithme}

\begin{algorithme}
  \fonction{estRempli}{o : Octet}{Boolean}
  {
  \retourner{nombreBit(o)=8}
  }
\end{algorithme}

\begin{algorithme}
  \fonction{obtenirBit}{o : Octet, pos :\naturel}{Bit}
  {
  temp : \naturel \\
  i : \naturel
  }
  {
  \affecter{temp}{getOctet(o)}
  \pour{i}{0}{8-pos-1}{}
    {
    \sialors{temp>=puissance(2,7-i)}
      {
      \affecter{temp}{temp-puissance(2,7-1)}
      }
    }
  \sialorssinon{temp>=puissance(2,pos)}
    {
    \retourner{bitA1}
    }
    {
    \retourner{bitA0}
    }
  }
\end{algorithme}

\begin{algorithme}
  \procedure{ajouterPoidsFaible}
  \paramEntreeSortie{o : Octet}
  \paramEntree{bit : Bit}
  {
  \sialors{bit=bitA1}
    {
    \affecter{o.octet}{o.octet + puissance(2,nombreBit(o))}
    }
  \affecter{o.nb}{o.nb+1}
  }

\end{algorithme}

\begin{algorithme}
  \procedure{ajouterPoidsFort}
  \paramEntreeSortie{o : Octet}
  \paramEntree{bit : Bit}
  {
  \sialors{bit=bitA1}
    {
    \affecter{o.octet}{o.octet + puissance(2,7-nombreBit(o))}
    }
  \affecter{o.nb}{o.nb+1}
  }

\end{algorithme}

\begin{algorithme}
  \fonction{octetEnDecimal}{o : Octet}{\naturel}
  {
  res : \naturel
  }
  {
  \affecter{res}{getOctet(o)}
  \retourner{res}
  }
\end{algorithme}

\begin{algorithme}
  \fonction{decimalEnOctet}{i : \naturel}{O_Octet}
  {
  res : Octet
  }
  {
  \affecter{res}{octetZero}
  \affecter{res.octet}{i}
  \sialors{i<2}
    {
    \affecter{res.nb}{1}
    }
  \sialors{2<=i ET i<4}
    {
    \affecter{res.nb}{2}
    }
  \sialors{4<=i ET i<8}
    {
    \affecter{res.nb}{3}
    }
  \sialors{8<=i ET i<16}
    {
    \affecter{res.nb}{4}
    }
  \sialors{16<=i ET i<32}
    {
    \affecter{res.nb}{5}
    }
  \sialors{32<=i ET i<64}
    {
    \affecter{res.nb}{6}
    }
  \sialors{64<=i ET i<128}
    {
    \affecter{res.nb}{7}
    }
  \sialors{128<=i ET i<256}
    {
    \affecter{res.nb}{8}
    }
  \retourner{res}
  }
\end{algorithme}

\begin{algorithme}
  \fonction{comparerOctet}{o1,o2 : Octet}{Boolean}
  {
  \retourner{nombreBit(o1)=nombreBit(o2) ET getOctet(o1)=getOctet(o2)}
  }
\end{algorithme}

\begin{algorithme}
  \fonction{octetEnCodeBinaire}{o : Octet}{CodeBinaire}
  {
  cb : CodeBinaire \\
  val,division,reste : \naturel
  }
  {
  \affecter{cb}{CodeBinaire()}
  \affecter{val}{octetEnDecimal(o)}
  \sialorssinon{val=0}
    {
    \instruction{ajouter(cb,bitA0)}
    }
    {
    \tantque{non(val/2 = 0)}
      {
      \affecter{division}{val/2}
      \affecter{reste}{val-2*division}
      \instruction{ajouter(cb,obtenirBit(decimalEnOctet(reste),0))}
      \affecter{val}{division}
      }
    \affecter{division}{val/2}
    \affecter{reste}{val-2*division}
    \instruction{ajouter(cb,obtenirBit(decimalEnOctet(reste),0))}
    }
  \retourner{cb}
  }
\end{algorithme}

\begin{algorithme}
  \fonction{codeBinaireEnOctet}{cb : CodeBinaire}{Octet}
  {
  res : Octet\\
  i : \naturel
  }
  {
  \affecter{res}{octetZero()}
  \pour{i}{0}{7}{}
    {
    \instruction{ajouterPoidsFaible(res,obtenirBit(cb,i+1))}
    }
  \retourner{res}
  }
\end{algorithme}
