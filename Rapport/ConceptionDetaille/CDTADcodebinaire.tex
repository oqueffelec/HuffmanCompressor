% auteur : Wacyk Jean-Gabriel
\begin{algolist}{0cm}

\type{CodeBinaire}{\^{}CB\_Noeud}
\begin{enregistrement}{CB\_Noeud}
  \item bit : Bit
  \item listeSuivante : \^{}CB\_Noeud
\end{enregistrement}\\

\end{algolist}

\begin{algorithme}
  \fonction{codeBinaire}{}{CodeBinaire}
  {}
  {
  \retourner{NIL}
  }
\end{algorithme}

\begin{algorithme}
  \procedure{ajouter}
  {
  \paramEntreeSortie{cb : CodeBinaire}
  \paramEntree{bit : Bit}
  }
  {
  pNoeud : CodeBinaire
  }
  {
  \affecter{pNoeud.bit}{bit}
  \affecter{pNoeud.listeSuivante}{cb}\\
  \affecter{cb}{pNoeud}
  }
\end{algorithme}

\begin{algorithme}
  \fonction{obtenirBit}{cb : CodeBinaire, pos : \naturel}{Bit}
  {}
  {
  \sialorssinon{pos=1}
    {
    \retourner{cb.bit}
    }
    {
    \retourner{obtenirBit(cb.listeSuivante,pos-1)}
    }
  }
\end{algorithme}

\begin{algorithme}
  \procedure{supprimerTete}
  {
  \paramEntreeSortie{cb : CodeBinaire}
  }
  {
  temp : CodeBinaire
  }
  {
  \affecter{temp}{cb}\\
  \affecter{cb}{cb.listeSuivante}
  \instruction{free(temp)}
  }
\end{algorithme}

\begin{algorithme}
  \procedure{supprimer}
  {
  \paramEntreeSortie{cb : CodeBinaire}
  }
  {}
  {
  \sialors{non(cb=NIL)}
    {
    \instruction{supprimerTete(cb)}
    \instruction{supprimer(cb)}
    }
  }
\end{algorithme}

\begin{algorithme}
  \fonction{longueur}{cb : CodeBinaire}{\naturel}
  {}
  {
  \sialorssinon{cb=NIL}
    {
    \retourner{0}
    }
    {
    \retourner{1+longueur(cb.listeSuivante)}
    }
  }
\end{algorithme}

\begin{algorithme}
  \fonction{copie}{cb : CodeBinaire}{CodeBinaire}
  {
  res : CodeBinaire\\
  i : \naturel
  }
  {
  \affecter{res}{codeBinaire()}
  \pour{i}{1}{longueur(cb)}{}
    {
    \sialorssinon{obtenirBit(cb,i)=bitA0}
      {
      \instruction{ajouter(res,bitA0)}
      }
      {
      \instruction{ajouter(res,bitA1)}
      }
    }
  \retourner{res}
  }
\end{algorithme}

\begin{algorithme}
  \fonction{compareCodeBinaire}{cb1,cb2 : CodeBinaire}{Boolean}
  {}
  {
  \sialorssinon{cb1=NIL ET cb2=NIL}
    {
    \retourner{TRUE}
    }
    {
    \sialorssinon{((cb1=NIL ET cb2!=NIL) OU (cb1!=NIL ET cb2=NIL))}
      {
      \retourner{FALSE}
      }
      {
      \sialorssinon{((obtenirBit(cb1,1)!=obtenirBit(cb2,1)))}
        {
        \retourner{FALSE}
        }
        {
        \retourner{compareCodeBinaire(cb1.listeSuivante,cb2.listeSuivante)}
        }
      }
    }
  }
\end{algorithme}
