%auteur : Quentin Robcis

\begin{algorithme}

  \type{ArbreDeHuffman}{\^{}ADH\_Noeud}
  \begin{enregistrement}{ADH\_Noeud}
    \item ponderation : unsigned int
    \item caractere : O\_Octet
    \item filsG : \^{}ADH\_Noeud
    \item filsD : \^{}ADH\_Noeud
  \end{enregistrement}\\

  \fonction{ADH\_arbreDeHuffman}{ponderation : unsigned int, caractere : O\_Octet}{ArbreDeHuffman}
  {
  arbre : ArbreDeHuffman
  }
  {
  \sialorssinon{arbre!=NIL}
  {
  \affecter{errno}{0}
  \affecter{arbre.ponderation}{ponderation}
  \affecter{arbre.caractere}{caractere}
  \affecter{arbre.filsG}{NIL}
  \affecter{arbre.filsD}{NIL}
  \retourner{arbre}
  }
  {
  \affecter{errno}{ADH\_ERREUR\_MEMOIRE}
  \retourner{NIL}
  }
  }\\
\end{algorithme}

\begin{algorithme}
  \fonction{ADH\_estUneFeuille}{arbre : ArbreDeHuffman}{\booleen}
  {
  }
  {
  \sialorssinon{((arbre.filsG)=NIL) et ((arbre.filsD)=NIL)}
  {\retourner{VRAI}}
  {\retourner{FAUX}}
  }\\
\end{algorithme}

\begin{algorithme}
  \fonction{ADH\_ajouterRacine}{arbre1 : ArbreDeHuffman, arbre2 : ArbreDeHuffman}{ArbreDeHuffman}
  {
  arbre : ArbreDeHuffman
  }
  {
  \sialorssinon{arbre!=NIL}
  {
  \affecter{errno}{0}
  \affecter{arbre.ponderation}{(arbre1.ponderation)+(arbre2.ponderation)}
  \affecter{arbre.caractere}{O\_OctetZero()}
    \sialorssinon{(arbre1.ponderation)<(arbre2.ponderation)}
    {
    \affecter{arbre.filsG}{arbre2}
    \affecter{arbre.filsD}{arbre1}
    }
    {
    \affecter{arbre.filsG}{arbre1}
    \affecter{arbre.filsD}{arbre2}
    }
  \retourner{arbre}
  }
  {
  \affecter{errno}{ADH\_ERREUR\_MEMOIRE}
  \retourner{NIL}
  }
  }\\
\end{algorithme}
